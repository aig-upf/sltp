\documentclass[a4paper]{article}
%\def\year{2020}\relax
%\usepackage{aaai20}  % DO NOT CHANGE THIS
\usepackage{times}  % DO NOT CHANGE THIS
\usepackage{helvet} % DO NOT CHANGE THIS
\usepackage{courier}  % DO NOT CHANGE THIS
\usepackage[hyphens]{url}  % DO NOT CHANGE THIS
\urlstyle{rm} % DO NOT CHANGE THIS
\def\UrlFont{\rm}  % DO NOT CHANGE THIS
\frenchspacing  % DO NOT CHANGE THIS
\setlength{\pdfpagewidth}{8.5in}  % DO NOT CHANGE THIS
\setlength{\pdfpageheight}{11in}  % DO NOT CHANGE THIS

% our packages
\usepackage{amsthm}
\usepackage{amsmath}
\usepackage{amssymb}
%\usepackage{bm}
\usepackage{booktabs}
%\usepackage[inline]{enumitem}
%\usepackage{mathtools}
%\usepackage{multirow}
%\usepackage[mode=buildnew,subpreambles=true]{standalone}
\usepackage{subcaption}
%\usepackage{todonotes}
\usepackage[table]{xcolor}  % TODO Comment out for final submission
%\usepackage{pgfplots}
%\usepackage{tikz}
\usepackage{natbib}


% Some useful macros
%\newcommand{\inlinecite}[1]{\citeauthor{#1}~(\citeyear{#1})}
%\newcommand{\citealp}[1]{\citeauthor{#1}~\citeyear{#1}}

\newcommand{\free}[1]{\ensuremath{\mathrm{free}(#1)}}
\newcommand{\vars}{\ensuremath{\mathrm{vars}}}
\newcommand{\pre}{\ensuremath{\mathrm{pre}}}
\newcommand{\add}{\ensuremath{\mathrm{add}}}
\newcommand{\del}{\ensuremath{\mathrm{del}}}
\newcommand{\effs}{\ensuremath{\mathrm{effs}}}

\newcommand{\tup}[1]{\ensuremath{\langle #1 \rangle}}
\newcommand{\tuple}[1]{\tup{#1}}  % Just an alias
\newcommand{\set}[1]{\ensuremath{\left\{#1 \right\}}}
\newcommand{\setst}[2]{\ensuremath{\left\{#1 \mid #2 \right\}}}

\newtheorem{definition}{Definition}
\newtheorem{definitionandtheorem}[definition]{Definition and Theorem}
\newtheorem{theorem}[definition]{Theorem}
\newtheorem{lemma}[definition]{Lemma}
\newtheorem{proposition}[definition]{Proposition}
\newtheorem{corollary}[definition]{Corollary}
\newtheorem{example}[definition]{Example}

\newcommand{\wip}[1]{{\color{red} #1}}  % From "work in progress" :-)
\newcommand{\gfm}[1]{\footnote{\color{red}{[Guillem] #1}}}

\newcommand{\numtasks}[1]{\small{(#1)}}


\setcounter{secnumdepth}{0} %May be changed to 1 or 2 if section numbers are desired.

%PDF Info Is REQUIRED.
% For /Author, add all authors within the parentheses, separated by commas. No accents or commands.
% For /Title, add Title in Mixed Case. No accents or commands. Retain the parentheses.
 \pdfinfo{
/Title ()
/Author ()
} %Leave this

\title{Representation Learning for Generalized Planning - SAT Encodings}
%Your title must be in mixed case, not sentence case.
% That means all verbs (including short verbs like be, is, using,and go),
% nouns, adverbs, adjectives should be capitalized, including both words in hyphenated terms, while
% articles, conjunctions, and prepositions are lower case unless they
% directly follow a colon or long dash
\author{GFM}


\begin{document}

\maketitle

\section{One-Shot Learning of Generalized Features \& Policy}

Encoding is parametrized by
\begin{itemize}
 \item $D$: maximum allowed value for $V(s)$ for any $s$ in the training set.
 \item max. concept complexity.
\end{itemize}


\subsection{Terminology}
A state is called \emph{reachable} if there is a path to it from $s_0$, and
it is called \emph{solvable} if there is a path from it to a goal
state.
A state is \emph{alive} if it is solvable, reachable and not a
goal state~\cite{frances-et-al-ijcai2019}.


\subsection{Variables}

\subsubsection{Main Variables}
\begin{itemize}
 \item $Good(s, s')$ for $s$ alive, $s'$ solvable.
 Note that if $s$ is alive but $s'$ is not solvable,
 it cannot make sense for a policy to recommend transition $(s, s')$.

 \item $Vleq(s, d)$, for $s$ alive, and $d \in [0, D]$.
 $Vleq(s, d)$ stands for $V(s) \leq d$.
 Note that for states $s$ that are a goal, we want $V(s)=0$, hance $V(s)\leq d$ for all $d$.
 Thus, we can restrict SAT variables $Vleq(s, d)$ to those states $s$ that are alive.

 \item $Select(f)$, for each feature $f$ in the feature pool.
\end{itemize}

\subsubsection{Auxiliary Variables}
\begin{itemize}
 \item $GV(s, s', d) \rightarrow Good(s, s') \land Vleq(s', d)$ for $s, s'$ alive, $d \in [0, D)$.
\end{itemize}

\subsection{Constraints}

\noindent The policy has to be complete with respect to all alive states:
\begin{align}
\bigvee_{s' \text{ solvable child of } s} Good(s, s'),&\;\; \text{for $s$ alive.}
\end{align}


\noindent Labeling a transition $(s, s')$ as ``good'' necessarily upper-bounds $V(s)$:
\begin{align}
 Good(s, s') \land V(s') \leq d \rightarrow V(s) \leq d+1,&\;\; \text{for $s, s'$ alive, $d \in [0, D)$.} \\
 Good(s, s') \rightarrow V(s) \leq d+1,&\;\; \text{for $s$ alive, $s'$ goal, $d \in [0, D)$.}
\end{align}

\noindent All transitions chosen as \emph{Good} need to have the same q-value:
\begin{align}
 V(s) \leq d+1 \land \neg V(s') \leq d \rightarrow -Good(s, s'),&\;\; \text{for $s, s'$ alive, $d \in [0, D)$.} \tag{\theequation${}^\prime$}
\end{align}


\noindent Any upper bound on $V(s)$ (for $s$ not a goal) needs to be justified:
\begin{align}
 V(s) \leq d+1 \rightarrow \bigvee_{s' \text{ goal child of } s} Good(s, s') \lor
                           \bigvee_{s' \text{ alive child of } s} GV(s, s', d),&
                           \;\; \text{for $s$ alive, $d \in [0, D)$.} \\
 \neg V(s) \leq 0,&\;\; \text{for $s$ not a goal.}
\end{align}

\noindent At least one atom $V(s) \leq d$ is true for each alive state $s$:
\begin{align}
 \bigvee_{d \in [0, D]} V(s) \leq d,&\;\; \text{for $s$ alive.}
\end{align}

\noindent Atoms $V(s) \leq d$ are consistent among them:
\begin{align}
 V(s) \leq d \rightarrow V(s) \leq d+1,&\;\; \text{for $s$ alive, $d \in [0, D)$.}
\end{align}

\noindent Good transitions can be distinguished from bad transitions:
\begin{align}
 Good(s, s') \rightarrow Good(t, t') \lor
 Distinguish(s, s', t, t'),&\;\; \text{for $s, t$ alive, $s', t'$ solvable.} \\
 Good(s, s') \rightarrow
 Distinguish(s, s', t, t'),&\;\; \text{for $s, t$ alive, $s'$ solvable, $t'$ unsolvable,}
\end{align}

\noindent where $Distinguish(s, s', t, t')$ is merely a presentational shorthand for $\bigvee_{f \in D2(s, s', t, t')} Select(f)$.


\bibliographystyle{plain}
% Cross-referenced entries need to go after the entries that cross-reference them
\bibliography{abbrv-short,literatur,references,crossref-short}
\end{document}
