\def\year{2019}\relax
%File: formatting-instruction.tex
\documentclass[letterpaper]{article} %DO NOT CHANGE THIS
\usepackage{aaai19}  %Required
\usepackage{times}  %Required
\usepackage{helvet}  %Required
\usepackage{courier}  %Required
\usepackage{url}  %Required
\usepackage{graphicx}  %Required
\frenchspacing  %Required
\setlength{\pdfpagewidth}{8.5in}  %Required
\setlength{\pdfpageheight}{11in}  %Required
%PDF Info Is Required:
  \pdfinfo{
/Title ()
/Author ()}
\setcounter{secnumdepth}{0}  

% \usepackage{soul}
% \usepackage[hidelinks]{hyperref}
% \usepackage[utf8]{inputenc}
% \usepackage[small]{caption}
%\usepackage{helvet}
%\usepackage{graphicx}
%\usepackage{url}

%\usepackage{color,soul}
%\newcommand\sr[1]{\sethlcolor{yellow}\hl{SR: #1}\sethlcolor{cyan}}
%\newcommand\hg[1]{\sethlcolor{green}\hl{HG: #1}\sethlcolor{cyan}}
%\newcommand\gdg[1]{\sethlcolor{yellow}\hl{GDG: #1}\sethlcolor{cyan}}
%\newcommand\bb[1]{\sethlcolor{cyan}\hl{BB: #1}}

\usepackage{amsmath}
\usepackage{amssymb}
\usepackage{amsthm}
\usepackage{enumerate}
%\usepackage{upgreek}
%\usepackage[normalem]{ulem}

%\usepackage[small]{caption}


\usepackage{xspace}
% \usepackage{ifthen}
\usepackage{tikz}
% \usetikzlibrary{shapes.geometric}
% \pgfdeclareimage[width=1cm]{dirt}{dirt2}

% \usetikzlibrary{shapes.geometric}
% \pgfdeclareimage[width=1cm]{dirt}{dirt2}

\newcommand{\Omit}[1]{}
\newcommand{\denselist}{\itemsep -1pt\partopsep 0pt}
\newcommand{\tup}[1]{\langle #1 \rangle}
\newcommand{\pair}[1]{\langle #1 \rangle}
% \newcommand{\citeay}[1]{\citeauthor{#1} [\citeyear{#1}]}
\newcommand{\citeay}[1]{\citeauthor{#1} (\citeyear{#1})}
\newcommand{\citeaywithnote}[2]{\citeauthor{#1} (\citeyear[#2]{#1})}

\newtheorem{definition}{Definition}
\newtheorem{definitionandtheorem}[definition]{Definition and Theorem}
\newtheorem{theorem}[definition]{Theorem}
\newtheorem{lemma}[definition]{Lemma}
\newtheorem{proposition}[definition]{Proposition}
\newtheorem{corollary}[definition]{Corollary}
\newtheorem{example}[definition]{Example}

\def\withproofs{0} % to include proofs in the draft: 0=don't include, 1=include
\newcommand{\nosuchproof}{\textcolor{red}{\bf No such proof yet...}}
\newcommand{\problem}{\textcolor{red}{\bf **** PROBLEM ****}}
\newcommand{\alert}[1]{\textcolor{red}{\bf #1}}
\newcommand{\CHECK}[1]{\textcolor{red}{\bf Note: #1}}

\newcommand{\hobs}{\ensuremath{h_{obs}}\xspace}
\newcommand{\indicator}[1]{[\![#1]\!]}
\newcommand{\C}{\mathcal{C}}
\renewcommand{\P}{\mathcal{P}}
\newcommand{\Q}{\mathcal{Q}}
\newcommand{\dexp}{\text{2-}\!\exp}

\newcommand{\Pn}{\ensuremath{P_n}\xspace}
\newcommand{\Pnm}{\ensuremath{P_{n,m}}\xspace}
\newcommand{\muwf}{\ensuremath{\mu_{w\!f}}\xspace}

\newcommand{\true}{\mathsf{true}}
\newcommand{\false}{\mathsf{false}}
\newcommand{\Eff}{{\mathit{Eff}}}

%% C++
\newcommand{\CC}{C\nolinebreak\hspace{-.05em}\raisebox{.4ex}{\tiny\bf +}\nolinebreak\hspace{-.10em}\raisebox{.4ex}{\tiny\bf +}}
%%% \def\CC{{C\nolinebreak[4]\hspace{-.05em}\raisebox{.4ex}{\tiny\bf ++}}}
%%% \newcommand{\pplus}{\!\hspace{-.05em}\raisebox{.4ex}{\tiny\bf +}\nolinebreak\hspace{-.10em}\raisebox{.4ex}{\tiny\bf +}}
%%% \newcommand{\mminus}{\!\hspace{-.05em}\raisebox{.4ex}{\tiny\bf -}\nolinebreak\hspace{-.10em}\raisebox{.4ex}{\tiny\bf - }}
%%%\newcommand{\pplus}{\hspace{-.05em}\raisebox{.2ex}{\footnotesize\mit +}\nolinebreak\hspace{-.10em}\raisebox{.2ex}{\footnotesize\mit +}}
%%%\newcommand{\mminus}{\hspace{-.05em}\raisebox{.2ex}{\footnotesize\mit -}\nolinebreak\hspace{-.07em}\raisebox{.2ex}{\footnotesize\mit -}}

% abstract actions and plans
\newcommand{\abst}[2]{\tup{#1;#2}}
\newcommand{\Rule}[2]{\ensuremath{#1 \Rightarrow #2}}

% decrements and increments
\newcommand{\pplus}{\hspace{-.05em}\raisebox{.15ex}{\footnotesize$\uparrow$}}
\newcommand{\mminus}{\hspace{-.05em}\raisebox{.15ex}{\footnotesize$\downarrow$}}

% numerical effects
\newcommand{\dec}[1]{#1:=#1-1}
\newcommand{\inc}[1]{#1:=#1+1}

% qualitative effects
\newcommand{\Dec}{Dec}
\newcommand{\Inc}{Inc}

% STRIPS actions
\newcommand{\Stack}{\text{Stack}}
\newcommand{\Unstack}{\text{Unstack}}
\newcommand{\Pickup}{\text{Pickup}}
\newcommand{\Putdown}{\text{Putdown}}
\newcommand{\Move}{\text{Move}}

% Examples
\newcommand{\Example}{\medskip\noindent\textbf{Example.}\xspace}


% \pdfinfo{ ..} 


\title{Learning features and abstract actions for computing generalized plans} 

% \author{B. Bonet, G. Frances, H. Geffner} % Alphabetical
% 
\author{\textbf{Submission \#5695}}


\begin{document}


\maketitle

\begin{abstract}
  Generalized planning is concerned with the   computation of  plans that solve multiple instances.
  Recently, it has been shown that such plans can be expressed as mappings of feature values into actions, and that they
  can often be  computed by means   of fully observable non-deterministic (FOND) planners after a sequence of
  transformations. The actions in such  plans, however,   are not the actions in the  instances themselves,  which  are
  not necessarily common to  other instances,   but  abstract actions that track the effect of the  actions
  on the given set of features. The formulation assumes  that both the features and the abstract actions are given.
  In this work, we address this limitation by  showing  how to learn them.  The resulting account of generalized
  planning  combines learning and planning in a novel way: a \emph{learner},  based on a min SAT formulation, yields
  the features and abstract actions  from    state transitions sampled  from some problem instances, 
  and a FOND \emph{planner} uses this information,   suitably   transformed as before,  to produce the general plans.
  Correctness guarantees are given and 
  experimental results on several  domains are reported.
\end{abstract}



\section{Introduction}

Generalized planning studies the computation of plans for  solving  multiple  instances at once
\cite{srivastava08learning,bonet09automatic,srivastava:generalized,hu:generalized,BelleL16,anders:generalized}.
For example, the plan  that iteratively  picks  a  clear block above $x$ 
and places it  on the table, achieves the goal $clear(x)$
in \emph{any} instance of the blocks world where the gripper is initially empty.
Once this  general plan or  policy is derived it can be  applied to solve an infinite collection
of instances that involve different initial states, different objects, and hence  different (ground)  actions. 


In the basic formulation due  to \citeay{hu:generalized}, a generalized plan
is a mapping of observations into actions that are assumed to be common
among  all the instances. More recently, this formulation has been extended by
\citeay{bonet:ijcai2018} to account for relational domains like Blocks world  where the set of
objects and  actions change from instance to instance.
For this,   the observations  are replaced by  a given  set of boolean
and numerical features $F$ and a set of \emph{abstract actions} $A_F$.
The  abstract actions are  \emph{sound} and \emph{complete} if they track the
effects  of the actions on  the features in a suitable way.
The resulting generalized plans  map feature values into abstract actions, and soundness
ensures that the application  of an   abstract action  results in the application of
a concrete action with a similar  effect  on  the features. Moreover, the form of the abstract actions
ensures that such plans can be computed using fully observable non-deterministic (FOND) planners
after some transformations.

\Omit{
Abstract actions can increase or decrease numerical variables  $n$
associated with the numerical features (e.g., number of blocks above $x$), and while
increments  are transformed into deterministic propositional  effects $n > 0$ where $n > 0$ is
the negation of the proposition $n=0$, decrements  are  transformed  into
non-deterministic (disjunctive)  propositional effects $n > 0 \, | \, n=0$. 
}

Bonet's and Geffner's  formulation of generalized planning, however, assumes that the
features and the abstract actions are given.   In this work, we address this limitation.
We achieving this by  \emph{showing  how features and abstract actions  can be learned from
the primitive predicates used to define the instances and state transitions
sampled from them.} For example,  in \cite{bonet:ijcai2018}, 
the general policy for achieving the goal $clear(x)$  is obtained 
using a FOND planner by introducing the boolean feature $H$ for holding a block,
the numerical feature $n(x)$ for counting the  number of blocks above $x$, and  two   abstract  actions:
one  with  preconditions  $\neg H$ and $n(x) > 0$, and effects $H$ and $n(x)\mminus$
(decrement $n(x)$), the other with precondition $H$ and effect $\neg H$.
In this work, we show how  to obtain such  policies without
having to provide the  features and  abstract actions by hand.
The abstraction is learned instead from the domain predicates
and  state transitions sampled from some  instances. 

The work has relation to a number of research threads in planning, knowledge representation
and machine learning.  We make use of SAT solvers and description logics for learning the features
and abstract actions \cite{sat-handbook,dl-handbook}. The abstract actions provide a model from which the plans are obtained via
transformations and FOND planners \cite{geffner:book,ghallab:book}.
The model is not the model of an instance but a generalized model for  obtaining  plans
that work for multiple instances. In this sense, the work is different than
action model learning \cite{yang:action-learning} and model-based reinforcement learning
and closer in aims to work on learning general policies from examples or experience
\cite{martin:concept,fern:bias,mazebase,general-drl}. Generalized planning has  also been formulated as a problem in first-order logic
\cite{srivastava:generalized} and general plans over finite horizons have also been derived using  first-order regression
\cite{boutilier2001symbolic,wang2008first,van2012solving}. Our approach differs from first-order approaches in the use
efficient propositional planners, and from purely learning approaches, in the \emph{formal guarantees} that are characteristic of planning:
if the learned abstract actions are actually sound,  the resulting general plan must be correct. 


The paper is organized as follows. We provide the background first and  show then  how features and
abstract actions can be  learned by enforcing soundness and completeness over a set of samples  using 
a  pool of features obtained the   domain  predicates. The full  computational model  is then
summarized,  following by the experimental results and a discussion.

\section{Background}

We review generalized planning, abstract actions, and computation of solutions 
following  \citeay{bonet:ijcai2018}.

\subsection{Generalized Planning}

A \emph{generalize planning problem} $\Q$ is a collection of planning instances $P$.
An instance $P$ is a  classical planning problems expressed in some compact language as a
tuple $P=\tup{V,I,A,G}$ where $V$ is a set of state variables that can take a finite set of values (boolean or not), $I$
is a set of atoms over $V$ defining an initial state $s_0$, $G$ is a set of atoms or literals over $V$ describing the goal states,
and $A$ is a set of actions $a$ with their preconditions and effects that define the set $A(s)$ of actions applicable
in any state $s$, and the successor state function $f(a,s)$, $a \in A(s)$. A state is a valuation over $V$, and a solution to
$P$ is an applicable action sequence $\pi=a_0,\ldots,a_n$ that generates  a state sequence $s_0,s_1,\ldots,s_{n}$
where $s_n$ is a goal state (makes $G$ true). In this sequence, $a_i \in A(s_i)$ and $s_{i+1}=f(a_i,s_i)$ for $i=0, \ldots, n-1$.
A state $s$ is reachable in $P$ if $s=s_n$ for one such sequence. A solution to the \emph{generalized problem} $\Q$ is
a solution to all instances $P \in Q$. The form of such solutions is described  below.

A \emph{feature} $f$ for a class $\Q$ of problems represents a function $\phi_f$
that takes an instance $P$ from $\Q$ and a state $s$ reachable in $P$, 
and results in a value denoted as $\phi_f(s)$. 
A feature is \emph{boolean}  if it results in a boolean value, and  \emph{numeric} if it results
in a numerical value, that we assume to be integer and non-negative. The features $H$
and $n(x)$ represent two features in the Blocks world (BW): the first a boolean
feature encoding that   the gripper is not empty; the second, a numerical
feature encoding  the number of blocks above  $x$. 

Symbols like $x$ to denote \emph{parameters} whose value depends on the instance  $P$ in  $\Q$.
For example, if $\Q_{clear}$  denotes  the BW instances with goals of the form $clear(x)$,
then the value of $x$ in an instance $P$ with goal $clear(a)$ is  $a$.

\subsection{Abstract Actions}

An \emph{abstract action}  for a generalized problem $\Q$ and a set of features $F$
is a pair $\bar{a}=\abst{Pre}{\Eff}$ where $Pre$ and $\Eff$ are the  preconditions
and effects expressed in terms of a  set $V_F$ of boolean and numerical \emph{state variables}
associated with the features in $F$. Preconditions and effects over boolean state variables $p$ are literals of the form $p$
and $\neg p$, that abbreviate the  atoms $p=true$ and $p=false$, while  over  numerical state variables $n$,
preconditions are of  the form $n=0$ and $n > 0$, and effects are  of the form $n\mminus$ (decrements) and
$n\pplus$ (increments).

Features $f$ refer to \emph{state functions} $\phi_f(s)$  in the instances of the generalized problem $\Q$,
but to \emph{state variables} in the abstraction.  Likewise, an \emph{abstract state} is  a truth valuation over the
state variables  $p$ associated with boolean features and the atoms $n=0$ over  state variables  $n$ associated with numerical features.
The abstract state $\bar{s}$  that \emph{corresponds} to a concrete state $s$ is the truth valuation that makes
$p$ true iff $\phi_p(s)=true$, and $n=0$ true iff $\phi_n(s) = 0$, and an abstract action $\bar{a}$
is applicable in a state $s$ if its preconditions are satisfied in $\bar{s}$.

A set of abstract actions $A_F$ is \emph{sound}  relative to a generalized problem $\Q$ 
iff for any reachable state $s$ over an instance $P$ in $\Q$, if an abstract action $\bar{a}$ in $A_F$
is applicable in $\bar{s}$, then there is a concrete action $a$ in $P$ that is applicable in $s$
that has the \emph{same qualitative  effects as}  $\bar{a}$ over the features. The action $a$
\emph{represents} or \emph{instantiates}  the abstract action $\bar{a}$ in $s$. 

Similarly, a set of abstract actions $A_F$ is \emph{complete}
iff for any reachable state $s$ over an instance $P$ in $\Q$, if a concrete action $a$ in $P$
that is applicable in $s$, then there is an abstract action $\bar{a}$ in $A_F$
that is  applicable in $\bar{s}$ that has the \emph{same qualitative effects over the features as $a$.}


The actions $a$ and $\bar{a}$ \emph{have the same qualitative  effects} in a state $s$ 
when  they have the same   effects on the boolean features $p$,  and the same \emph{qualitative effects} on the numerical
features $n$; i.e., $a$ decreases (resp. increases) the value of the function $\phi_n(s)$ in $s$ iff
$\bar{a}$ decreases (resp. increases) the value of the state variable  $n$.

\begin{example}
  For the features $F=\{H,n(x)\}$, the abstract action  $\bar{a}=\abst{\neg H,n(x)>0}{H,n(x)\mminus}$ 
  is sound in  problem $\Q_{clear}$ that includes the BW instances with stack and unstack actions.
  Indeed, if $s$ is a reachable state in an instance $P$ and $\bar{a}$ is applicable in $\bar{s}$, 
  the gripper must be empty in $s$ and some block  must be above $x$. If $A$ is the clear block above $x$ and $on(A,B)$ is true in $s$,
  then the  action $a=unstack(A,B)$  will be  applicable in $s$ and have the same qualitative effects of  $\bar{a}$
  over $F$: they both make $H$ true and decrease the value of $n(x)$. The abstract action $\bar{a'}=\abst{H}{\neg H}$ is also sound.
\end{example}

For a set of abstract actions to be complete in this example, it should 
should contain  abstracts that track the  effects of the other  actions
on the features such as stacking a block above $x$, picking up a block that is not above $x$, and so on.
It may well be possible that a particular feature set $F$ can't support a set of abstract actions
$A_F$ that is both sound and complete relative to  $\Q$. Soundness, however,  is crucial for deriving general plans as it ensures
that in any state of any instance of $\Q$,  an applicable  abstract action  can be ``instantiated'' 
by   one or more concrete actions that have the same qualitative effects on the features. 

\subsection{Solutions}

A solution to a  generalized problem $\Q$ given the  features $F$ and  abstract actions $A_F$
is a partial  function $\pi$ that maps \emph{abstract states} into \emph{abstract actions} such that
$\pi$ solves all instances $P$ in $\Q$. The plan or policy $\pi$ induces a  trajectory $s_0,a_0,s_1, \ldots, s_n$ in an instance $P$ of $\Q$
iff a)~$s_0$ is the initial state in $P$, b)~$a_i$ is one of the actions that instantiate 
$\pi(\bar{s_i})$ in $s_i$, c)~$s_{i+1}=f(a_i,s_i)$ if $a_i$ is applicable in $s_i$,
and d)~$s_n$ is a goal  state of $P$,  $n=\infty$, or $\pi(\bar{s_n})$ is undefined, denotes an abstract action
that is not applicable in $\bar{s_n}$, or no applicable action $a_n$ instantiates  $\pi(\bar{s_n})$.
The policy $\pi$ solves $P$ iff all the induced trajectories reach the goal of $P$.

\begin{example}
A solution for the generalized problem $\Q_{clear}$  with feature set $F=\{H,n(x)\}$
is the policy $\pi$  given by the rules $\Rule{\neg H, n(x)>0}{\bar{a}}$ and $\Rule{H, n(x)>0}{\bar{a}'}$
for the abstract actions $\bar{a}$ and $\bar{a'}$ above. The policy picks blocks above $x$ and puts them aside
(not above $x$) until $n(x)$ becomes zero.
\end{example}

\subsection{Computation}

The steps for obtaining general policies $\pi$ for a generalized problem $\Q$
using the features $F$ and the  abstract actions $A_F$ are as follows \cite{bonet:ijcai2018}:

\medskip

\noindent 1. The state variables $V_F$ and the abstract actions $A_F$ are extended
  with initial and goal  formulas $I_F$ and $G_F$  over $V_F$ to yield a   \emph{numerical, non-deterministic planning problem}
  $Q_F=\tup{V_F,I_F,G_F,A_F}$. For soundness,  $I_F$ must be  such that the initial states of instances $P$ in $\Q$ all satisfy $I_F$,
  while all states that satisfy $G_F$ must be goal states of $P$.\footnote{
   A state $s$ is said to satisfy $I_F$ or $G_F$ if  $\bar{s}$ does.}

  
\medskip

\noindent 2. The problem $Q_F=\tup{V_F,I_F,G_F,A_F}$ is converted into a \emph{boolean FOND}
  problem $Q'_F=\tup{V'_F,I'_F,G'_F,A'_F}$ by replacing 
  the numerical variables $n\in N$ by the symbols $n=0$, the first-order literals $n=0$ by propositional literals  $n=0$, 
  the effects $n\pplus$ by deterministic effects $n > 0$, and the effects $n\mminus$ by non-deterministic effects $n>0\,|\,n=0$.

  \medskip
  
\noindent 3. The strong cyclic solutions of  $Q'_F$ assume that the
  non-deterministic effects $n>0\,|\,n=0$ are \emph{fair} \cite{strong-cyclic},  
  yet  they are  \emph{conditionally fair} only: infinite decrements of $n$ ensure $n=0$  eventually but just without
  infinite  increments. The problem of obtanining  solutions of $Q'_F$ that assume conditional fairness only, the  terminating
  solutions in   \cite{srivastava:aaai2011}, is mapped into the problem of solving an  amended FOND $Q^+_F$
  \cite{bonet:ijcai2017}.\footnote{For this, we use a translation made available by the authors that fixes bugs in the  description found in the paper
     at https://github.com/bonetblai/qnp2fond.}

\medskip

The strong cyclic solutions of  $Q^+_F$ computed by FOND planners off-the-shelf   are solutions to the generalized problem $\Q_F$. 

\begin{example}
  Let restrict  $\Q_{clear}$  to those  instances where  the gripper is initially empty
  and there are blocks on top of $x$. For $F=\{H,n(x)\}$  and $A_F=\{\bar{a},\bar{a'}\}$ as above, 
   $Q_F=\tup{V_F,I_F,G_F,A_F}$ can be defined with $I_F=\{\neg H, n(x) > 0\}$ and $G_F=\{n(x)=0\}$.
   $Q'_F$ is like $Q_F$ but with $n(x)=0$ regarded as a propositional symbol, $n(x) > 0$
  as its negation, and the effect $n(x)\mminus$ of  $\bar{a}$ replaced by  $\neg n(x)=0 \, | \, n(x)=0$.
  Since  no action in $A_F$  increases $n(x)$,   $Q^+_F=Q'_F$, and the strong cyclic solution  of $Q'_F$,
  that corresponds to the policy above, solves $\Q_{clear}$.
\end{example}


\section{Approximate Soundness and Completeness}


Bonet's and Geffner's formulation and computational model rest on 
features $F$ and  sound abstract actions $A_F$ provided by hand.
The contribution of this work is to learn them both from samples:


\begin{definition}
 For a generalized problem $\Q$, a \emph{sample set}  $\cal S$ is a non-empty set of tuples  $(s,s',P)$
 such that 1)~$P$ is an instance of $\Q$, 2)~$s$ is a reachable state in $P$, 3)~$s'$
 is a possible succesor of $s$ in $P$; i.e., $s'=f(a,s)$ for some action $a$ applicable in $s$,
 4)~$\cal S$ is closed in the sense  that if $(s,s',P)$ is in $\cal S$, then $(s,s'',P)$ is in $\cal S$  if $s''$ is another possible succesor of $s$ in $P$.
\end{definition}

By assuming that sampled states $s$ are tagged with the instance $P$, 
we can abbreviate the tuples $(s,s',P)$ as $(s,s')$, and refer to
the states $s'$ in  the pairs $(s,s') \in {\cal S}$ as the successors
of $s$ in ${\cal S}$. Notice that the closure condition 4) ensures that
states $s$ appearing \emph{first} in pairs $(s,s')$ must be fully expanded in ${\cal S}$;
namely, all possible transitions $(s,s'')$ must be in the sample as well. We call them
the \emph{expanded} states in ${\cal S}$.

The features and abstract actions for $\Q$ are  learned by
enforcing an approximate form of soundness and completeness
over the samples. Like before,  we take a transition $(s,s')$ and an abstract action $\bar{a}$
to have  the same qualitative effects over a boolean  feature  iff they both make it true , they both make it false,  or
they both leave it unchanged,  and over a numerical feature  iff they both decrease it,  they both increase it,  or they both leave it unchanged.

\begin{definition}
  A set of abstract actions $A_F$ is sound \emph{relative to a  sample set} $\cal S$ for $\Q$
  iff for  any  abstract action $\bar{a}$ in $A_F$ that is applicable in  a expanded state $s$ of ${\cal S}$,
  there is a transition $(s,s')$ in $\cal S$ that has the same qualitative  effects over $F$ as $\bar{a}$.
\end{definition}

\begin{definition}
A set of abstract actions $A_F$ is complete   \emph{relative to a sample set} $\cal S$ for $\Q$
iff for each transition  $(s,s')$ in  $\cal S$, there is an abstract action $\bar{a}$ in $A_F$
with the same qualitative effects over in $F$ that is applicable in $s$.
\end{definition}

For a sufficiently large sample set, these approximate notions of soundness and completeness
converge to the exact notions. 

\section{Learning  features and  abstract actions}

For computing the features and abstract actions
we define  a  formula  $T({\cal S,F})$ that  is \emph{satisfiable} 
iff there is a set of features $F  \subseteq {\cal F}$ and a set of abstract actions $A_F$ over $F$
that is sound and complete relative to the sample set ${\cal S}$.

\Omit{The  sets  $F$ and $A_F$ can
     then be read off  from any  satisfying assignment. Later we will add weights
     to so
     me of the clauses to establish a preference over these assignments
     (e.g., for  minimizing $|F|$).}

\subsection{SAT Encoding}


For each transition  $(s, s') \in \mathcal{S}$ and each feature $f \in \mathcal{F}$,
$\Delta_f(s, s') \in \{+, -, \uparrow, \downarrow, \bot\}$ will denote the \emph{qualitative change of value}
of feature $f$ along the transition $(s, s')$, which can go from false to true ($+$), from true to false ($-$) or remain unchanged ($\bot$),
for boolean features, and can increase ($\uparrow$), decrease ($\downarrow$), or remain unchanged ($\bot$), for numerical features.
The \emph{propositional variables}  in $T({\cal S,F})$ are then:

  \begin{itemize}
  \item  $selected(f)$ for each $f \in {\cal F}$,  true iff  $f$  selected  (in  $F$).
  \item $D_1(s,t)$ for  states   $s$ and $t$ expanded in ${\cal S}$,  true iff selected features distinguish  $s$ from $t$;
      i.e.  if $s$ and $t$ disagree on the truth value of a selected boolean feature $p$ or atom $n=0$ for a selected numerical feature $n$.
      % Since $D_1$ is symmetric, only symbols for  $s < t$ are created   for  some  arbitrary, fixed order over the states in ${\cal S}_1$.
  \item $D_2(s, s', t, t')$ for  each $(s, s')$ and $(t, t')$ in $\mathcal{S}$,
    true iff selected features $f$ distinguish the two transitions; i.e., if $\Delta_f(s, s')\not=\Delta_f(t,t')$. 
%      As for $D_1$, the number of $D_2$ variables is cut in half due to symmetries.
  \end{itemize}

\medskip


\noindent The  first  formulas in $T({\cal S,\cal F})$ capture the meaning of $D_1$:
  
\begin{equation}
  D_1(s, t) \Leftrightarrow \bigvee_{f} \, selected(f)
  \label{d1}
\end{equation}

\noindent where $f$ ranges over features in $\mathcal{F}$ with  different qualitative values in $s$ and $t$; i.e.
boolean features $p$ with different truth value in $s$ and $t$, and numerical features $n$ with different truth values
for atom $n=0$ in $s$ and $t$. 

\noindent The second formulas encode  the meaning of $D_2$:

\begin{equation}
  D_2(s, s', t, t') \Leftrightarrow \bigvee_f \,  selected(f)
\end{equation}

\noindent where $f$  ranges over features $f \in \mathcal{F}$ that have the same qualitative values in $s$ and $t$
but which change differently in the two transitions; i.e.,  $\Delta_f(s, s') \neq \Delta_f(t, t')$.

The third  class of formulas relate $D_1$ and $D_2$ by  enforcing  \emph{soundness} and \emph{completeness} over  the samples in ${\cal S}$.
Yet, since the abstract actions  are not given but are derived, the qualitative changes in each  transition $(s,s')$ in ${\cal S}$
are  taken as templates  of abstract actions. Thus, if $s$ and $t$ are not distinguished by the selected features, 
for each transition $(s, s') \in \mathcal{S}$, there must be a transition $(t, t') \in \mathcal{S}$
such that that the two transitions are not distinguished by the selected features either.
This is expressed as

\begin{equation}
  \neg D_1(s, t) \, \Rightarrow \, \bigvee_{t'} \neg D_2(s, s', t, t')
  \label{bridge1}
\end{equation}

\noindent where $s$ and $t$ are expanded states in  ${\cal S}$, $s'$ is a successor of $s$ in ${\cal S}$,
and $t'$ ranges over the successors of $t$ in ${\cal S}$.

%%% --
\Omit{ % left out: this is for efficiency only, and is not strictly needed. Also, doesn't fit as such with goal-marking
  Refer to this as an ``efficiency optimization'' ..
  
  such that $s < t$, and transition $(s,s')$ in ${\cal S}$. Due to the symmetry breaking caused by the constraint $s < t$,
  the following formulas are also needed

\begin{equation}
  \neg D_1(s, t) \, \Rightarrow \, \bigvee_{s'} \neg D_2(s, s', t, t')
    \label{bridge2}
\end{equation}

\noindent where the  iteration in the right hand disjunction goes over the different successor states $s'$ of the first argument
$s$ of $D_1$ that are in the sample set ${\cal S}$, $s$ and $t$ are states in ${\cal S}_1$ as before with $s < t$, and $t'$ is a
successor state of $t$ in ${\cal S}$; i.e. $(t,t') \in {\cal S}$.
}
%% --


The last  set of clauses force the selected features to distinguish goal from non-goal states as:

\begin{equation}
  D_1(s,t) 
\label{goal}
\end{equation}

\noindent where  $s$ and $t$  are expanded states  ${\cal S}$ such that exactly one of them is a goal state. 
For this, it is assumed that  states $s$ in   tuples $(s,s',P)$ are labeled as   goal states  of $P$ or non-goals.

The theory $T({\cal S},{\cal F})$ given by the formulas  \ref{d1})--(\ref{goal}
has a number of atoms that is linear in the size of the feature pool $\cal F$,
and quadratic in the number of expanded  states $s$  and transitions $(s,s')$ in ${\cal S}$.
The resulting number of clauses is ....  , which can be roughly cut in half by  symmetry considerations.

In the experiments we will   consider also  a smaller theory  that avoids the consideration of all pairs
of expanded states or all pairs of transitions in $\cal S$]. For this, some expanded states $s$ and transitions $(s,s')$ in
tuples $(s,s',P)$ are marked as \emph{goal relevant}. This is achieved by computing some plans for some instances $P$ in $\Q$, 
and by adding the resulting state transitions to ${\cal S}$ at the same time that the first
arguments $s$ and $(s,s')$ in the  $D_1$ and $D_2$ atoms are restricted to be marked.
If the set of marked states and transitions is much smaller than the set of unmarked ones, 
then the resulting formula,  $T_G({\cal S},{\cal F})$ will be much smaller than $T({\cal S},{\cal F})$.
The goal marking scheme that involves the theory $T_G({\cal S},{\cal F})$ in place of $T({\cal S},{\cal F})$
amounts to enforcing soundness on  the sample set ${\cal S}$ but not completeness, yet this does
not affect the resulting  formal guarantees that depend on soundness only. 



\subsection{Extracting $F$ and  $A_F$}


If  $\sigma$ is  a satisfying assignment of the theory $T({\cal S},{\cal F})$, 
we define $F_{\sigma}$ as the set of  features $f$ in $\cal F$ such that $selected(f)$ is true in $\sigma$,
and $A_{\sigma}$ as the set of abstract actions that capture  the  transitions $(s,s')$ in ${\cal S}$
with duplicates and redundancies removed. The abstract action $\bar{a}=\abst{Pre}{\Eff}$
that captures the transition $(s,s')$ has the \emph{precondition} $p$ (resp. $\neg p$)
if $p$ is a boolean feature in $F_{\sigma}$ that is true (resp. false) in $s$, and has
the precondition $n=0$ (resp. $n > 0$) if $n$ is a numerical feature in $F_{\sigma}$
such that $n=0$ is true (resp. false) in $s$. Similarly, $\bar{a}$ has the \emph{effect}
$p$ (resp. $\neg p$) if $p$ is a boolean feature in $F_{\sigma}$ that is true (resp. false)
in $s'$ but false (resp. true) in $s$, and the effect $n\mminus$ (resp. $n\pplus$) if $n$ is a numerical feature in $F_{\sigma}$
whose values decreases (resp. increases) in the transition from $s$ to $s'$. 
If two abstract actions $\bar{a}$ and $\bar{a}$ differ only in the sign of a precondition,
the precondition can be dropped, removing the duplicates and iterating until  no more simplifications
can be done. They key properties are:


\begin{theorem}
  The theory $T({\cal S},{\cal F})$ is satisfiable iff there is a set of features $F \subseteq {\cal F}$
  and a set of abstract actions $A_F$ over $F$ such that $A_F$ is sound and complete relative to  ${\cal S}$.
\end{theorem}

\begin{theorem}
  If $\sigma$ is a satisfying assigment of $T({\cal S},{\cal F})$, then the abstraction actions in $A_{\sigma}$ 
  are  sound and complete  relative to  ${\cal S}$.
\end{theorem}

\begin{theorem}
  If $\sigma$ satisfies  $T_G({\cal S},{\cal F})$, then $A_{\sigma}$   are  sound  relative to  ${\cal S}$.
\end{theorem}

\Omit{
%% commented out
The goal expression $G_F$ can be provided explicitly  in terms of the primitive predicates,
but  more generally can be extracted  from the satisfying assigment $\sigma$ as well.
For this, $G_F = G_{\sigma}$ is defined as the DNF formula whose terms 
correspond to the abstract states over the selected features in $F_{\sigma}$ 
that are false in all expanded states in ${\cal S}$ marked as non-goals
and true in some expanded goal state. This DNF formula can   be simplified in standard ways. 
Due to  (\ref{d1}) and (\ref{goal}) that force the selected features in $F_{\sigma}$
to   distinguish goal  from non-goal states, we have that:

\begin{theorem}
For a satisfying assigment $\sigma$  of $T({\cal S},{\cal F})$, 
$s$ is expanded goal   state in ${\cal S}$  iff  $s$ satisfies  $G_F=G_{\sigma}$.
\end{theorem}
}


Different  satisfying  assignments $\sigma$,  give rise to different  features $F_{\sigma}$
and abstract actions $A_{\sigma}$. More meaningful features are found by looking for
satisfying assignments $\sigma$  that minimizes a cost measure such as $|F_{\sigma}|$
or  more generally $\sum_{f \in F_{\sigma}} cost(f)$. We will  achieve this  constrained
optimization through the use  Min SAT or Max-Weighted SAT solvers,  where it suffices to give
infinite weights  to the clauses resulting from $T({\cal S},{\cal F})$ and weights given by $cost(f)$ 
to the ``soft''  unit clauses  $\neg selected(f)$.

\section{Feature Pool}

The computation of the feature pool $\cal F$ used in $T({\cal S},{\cal F})$
relies on the predicates used to encode the instances in $\Q$.
A simple grammar   allow  us   to construct new predicates from them, 
and the boolean and numerical features in the pool
are defined from the resulting \emph{unary predicates}. 
The key issue is how to define new predicates from old  ones. For this 
we take a solution off-the-shelf: \emph{description logics (DL)},  called  also
\emph{concept languages},  are grammars for doing exactly this.
Concepts $C$  refer  to predicates of arity 1 and roles $R$  to predicates of arity 2.
Concept languages have been used before for  generating general policies using purely learning methods  \cite{martin:concept,fern:bias}.
For details on the syntax and semantics of description logics; see \cite{dl-handbook}.

\subsection{Grammar}

The DL grammar is given by the rules:

  \begin{itemize}
  \item $C \ \leftarrow\ \ C_p, C_u, C_x$, primitive, universal,  nominal c's,\footnote{Denotation of $C_x$ for parameter $x$ is $\{x\}$.}
  \item $C \ \leftarrow\  \neg C, C \sqcap C'$, negation and conjunction, 
  \item $C \ \leftarrow\  \exists R.C , \forall R.C$, concepts that use of  roles: first denotes $x$ s.t. for some  $y$,  $R(x,y) \land C(y)$.
  Second,  $x$ s.t. for all $y$, $R(x,y) \rightarrow C(y)$,
  \item $C \ \leftarrow\  R=R'$, stands for $x$ s.t. set of $y$'s  s.t. $R(x,y)$ equal to set of $y$'s s.t. $R'(x,y)$,
    \item $R \leftarrow R_p, R^{-1}, R^*$ ; primitive, inverses, transitive.
%     \item $R \leftarrow R \circ R'$, role composition: $(x,y)$ in $R \circ R'$ if $(x,z)$ in $R$ and $(z,y)$ in $R'$.
%     \item $R \leftarrow R:C$, denotes pairs $(x,y)$ in $R$ s.t. $C(y)$.
 \end{itemize}

The  compositional  semantics   of these rules allows us to determine the denotations (extensions)   $C^s$ and $R^s$  of
the resulting unary and binary predicates from the denotation of the primitive predicates in any  state $s$. 

** Check what is used here and not; see what needs to be explained; keep concise, improve lay out.


\subsection{Pool}


The set of  concepts and roles  with complexity bounded by $k$ is denoted as ${\cal G}(k)$
with their  complexity given by the    minimum number of  rules for generating them.
The size  of this set is reduced by incrementally pruning ``duplicate'' concepts,
those whose denotation coincides with the denotation of a another concept over all
the sampled states in ${\cal S}$. 

The pool of features ${\cal F}={\cal F}^k$  is defined from   ${\cal G}^k$
and contains the  features $f$ that represent the functions $\phi_f(s)$
as follows:

\begin{itemize} 
\item Numerical   $n_C$ for each $C$  that  stand for cardinality $|C^s|$,
\item Boolean  $b_C$ for each   $C$ that are  true in $s$ iff $|C^s| > 0$,
\item Boolean   $b_p$ for each primitive predicate $p$ of arity $0$,
\item Numerical $\textit{dist}(C_1,R:C,C_2)$,  that represent that smallest $n$ such that for $x_1, \ldots, x_n$, 
$C_1^s(x_1)$, $C_2^s(x_{n})$, and $R:C(x_i,x_{i+1})$ for $i=1, \ldots, n$. 
\end{itemize}  


The role $R:C$ is such that its denotation contains the pairs $(x,y)$ in $R$ such that $y$ is in $C$.
Some pruning is done in this space of features as well. E.g., numerical features $n_C$ are pruned from the pool
when there are  no states $s$ and $s'$ in the sample set for which $n_c(s)=0$ and $n_c(s') > 1$,
while boolean features $n_b$  are pruned if uniformly true or false for all $s$, or if $n_C$ not pruned. 
The measure  $cost(f)$ is set to the complexity of $C$ for $n_C$ and $b_C$, to $0$ for $b_p$
and to the complexities of $C_1$, $R$, $C$, and $C_2$ added up for $\textit{dist}(C_1,R:C,C_2)$.
Only features with cost bounded by $k$ are allowed in $\cal F$.

\section{Computational Model: Summary}

In summary, the steps for computing general plans are:

\begin{enumerate}
\item a sample set ${\cal S}$ is computed from a few  instances $P$ of  a given generalized problem $\Q$ 
\item a pool of features $\cal F$ is obtained from the predicates used in the instances $P$, the  grammar, and bound $k$,
\item a satisfying assignment $\sigma$ of the theory $T({\cal S},{\cal F})$ or $T_G({\cal S},{\cal F})$
that minimizes  $\sum_{f \in F_{\sigma}} cost(f)$ is obtained using a Weighted-MAX SAT solver,
\item the features $F$ and abstract actions $A_F$  are extracted from $\sigma$ as $F_{\sigma}$ and $A_{\sigma}$,
\item the problem $Q_F=\tup{V_F,I_F,G_F,A_F}$ is  defined with initial and goal conditions $I_F$ and $G_F$ provided by hand
to match $\Q$, 
\item the FONDP  $Q^+_F$, constructed from $Q_F$,  is solved by a FOND planner off-the-shelf
\end{enumerate}

Note that the  policy $\pi$ that results from the last step  deals with propositional symbols
that correspond to atoms $p$ and $n=0$ for boolean and numerical features $p$ and $n$ in $F$. 
When this policy is applied to an instance $P$ of $\Q$, however, such features denote state functions
which are determined by the  semantics of the concepts and roles used in  the features.
(e.g., $n_C$ stands for feature function $\phi_{n_c}(s)=|C^s|$).
The exception are the boolean features $b_p$ associated with
predicates $p$ of zero arity which are treated as fluents.

If we write $A \models B$ to express that all the states that satisfy $A$, satisfy $B$, 
then the formal guarantees can be expressed as:

\begin{theorem}
If the abstract actions $A_F$ that are sound relative to the sample set $\cal S$  are actually sound,
then a   policy $\pi$ that solves  $Q^+_F$  is guaranteed to solve  all instances $P$ of $\Q$
with initial and goal situations $I$ and $G$ s.t. $I \models  I_F$ and $G_F \models G$.
\end{theorem}

This means that a  policy $\pi$ obtained with a FOND planner from $Q^+_F$ 
solves solves the  generalized problem $\Q$, provided that the  abstract
actions learned  are  sound and  that $\Q$ is restrictedto the instances
$P$ whose initial and goal conditions comply with the conditions on $I_F$ and $G_F$.
Since we don't have a language for specifying $\Q$ formally, we can take the formulas
$I_F$ and $G_F$ indeed as a partial formal specification of $\Q$. The theorem then says that
if soundness over the samples, ensure formal soundness over $\Q$, the policy that solves the
FONDP $Q^+_F$, solves $\Q$. The result holds whether $T$ or $T_G$ (goal marking) is used in step 3.
In principle, we could learn the formulas $I_F$ and $G_F$ from the sample as well,  but with different guarantees. 

\section{Experimental Results}

We evaluate the computational model given by steps 1--6 above over several domains. *Fill up general  details in 1--6
that don't change from problem to problem: how ${\cal S}$ obtained, value of $k$, SAT solver used, FOND planners, translator, etc.
The goal marking scheme that  yields the smaller  theory $T_G({\cal S},{\cal F})$ is obtained an optimal plan over
a few (always less than xxxx) instances $P$. 

Experiments setting: cpu, cutoffs, etc.

For each problem: 1--6 details that vary from problem to problem. Follow same structure:
$\Q$, $P$s, primitive predicates, number of atoms and clauses  of $T$ and/or $T^G$,
SAT times,  $F$, $|A_F|$;  number of FOND atoms and actions in $Q^+_F$ (include those in $F$, $A_F$ plus translation), $I_F$ and $G_F$,
FOND time, resulting policy.  Numbers can go all in a table.

** Mention: symbold DRL paper by Shanahan in 

\section{Discussion}

We have introduced a scheme for computing general plans that mixes learning and planning: a learner infers
a general abstraction made up of features and abstract actions by enforcing  a notion of soundness and completeness
over a suitable set of samples, which is transformed and fed into a FOND planner.
The number of samples required for obtaining correct general plans is very small,
as the learner doesn't have to learn how to plan,  but just to identify the
general features that are relevant for the planner. Moreover, unlike, purely
learning approaches, the scope and correctness of the resulting general plans
can be assessed.  We have carried out a number of experiments over several
domains illustrating the versatility of the approach. In the future, we
want to explore the use of the learned  features and actions
in classical planning, the study of  formal characterizations
of the generalized problem $\Q$ and  the possibilities
of testing soudness over $\Q$ automatically, and ways
for improving the scalability of the approach, for example,
by collecting samples intelligently and incrementally,
from the states where the computed  policy fails,
and by reusing the features and actions obtained
for some goal for more complex goals.

\Omit{


** discussion  of experiments; things we do well and not, scope, lessons. limitations (expressive? scalability?)

** briefly related work, expanding paragraph in intro.

** summary and cconlusions
}

\bibliographystyle{aaai}
\bibliography{control}

\end{document}

