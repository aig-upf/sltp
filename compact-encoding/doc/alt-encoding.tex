\documentclass[10pt]{article}

\usepackage{amsmath,amssymb,amsthm}
\usepackage[margin=1in]{geometry}
\usepackage{enumerate}
\usepackage{booktabs}
\usepackage{xspace}

\newcommand{\Omit}[1]{}
\newcommand{\tup}[1]{\langle #1 \rangle}
\newcommand{\denselist}{\topsep -1pt \itemsep -1pt}

\newcommand{\A}{\mathcal{A}}
\newcommand{\E}{\mathcal{E}}
\newcommand{\F}{\mathcal{F}}
\newcommand{\G}{\mathcal{G}}
\newcommand{\M}{\mathcal{M}}
\renewcommand{\S}{\mathcal{S}}

\newcommand{\pre}[1]{\ensuremath{{\scriptsize\bullet}#1}}
\newcommand{\post}[1]{\ensuremath{#1{\scriptsize\bullet}}}
\newcommand{\act}[1]{\ensuremath{a(#1)}}

\newtheorem{definition}{Definition}

\begin{document}
\allowdisplaybreaks


\begin{center}
\Large\bf
Alternative encoding for learning feature-based reformulations \\
  {\normalsize Version 22/07/2018}
\end{center}


\bigskip\noindent
We aim at a reduction to SAT with parameters $K$ and $N$: $K$ bounds the size of
the abstract space by selecting $K$ features from the set of available features,
and $N$ bounds the number of abstract actions that can be used in the abstract
space.

The general idea is to use ``meta-variables'' for selecting the features that
define the abstract state model and then using ``meta-variables'' that define
the abstact actions over the abstract states.
The SAT problem ``constructs'' the abstract model and then verifies soundness
and completeness with respect to the transitions and states in the sample.

The set of available features is denoted with $\F$ and the set of sampled
transitions as $\M$.  The value of features in $\F$ at the states in the sample
$\M$ are given as input to the reduction in the form of a matrix $M$ that has
as many rows as the number of features in $\F$ and as many columns as the
number of states in $\M$.
We denote with $S$ and $T$ the set of states and transitions in $\M$.

If the unique states in $\M$ are $\{s_0,s_1,\ldots\}$, we use integer
$i$ to denote state $s_i$.
For boolean features $f$ and state $s_i$, the entry $m_{fi}$ in the matrix
$M$ is 1 iff the feature $f$ holds at state $s_i$.
For numerical features $f$ and state $s_i$, the entry $m_{fi}$ is the
value of $f$ at state $i$.

\bigskip\noindent
The size of the encoding is summarized in the table below where $S$ is number
of states, $T$ is number of transitions, $F$ is number of features, $L=\log_2(F)$,
$\Delta\leq L$, $D$ and $U$ are numbers for dummy and inexistent features
respectively, and $Q(L,K,N)=N(1+4K) + K(D+U) + (K-1)(L^2 + L) - 1$.

\begin{center}
  \begin{tabular}{lc}
    \toprule
      Parameters:   & $K=\text{\# selected features}$, $N=\text{\# abstract actions}$ \\
      \# variables: & $2SK(1 + N) + T(2K + 3N) + 2K(2N + L) - L$ \\
      \# clauses:   & $S(2KF + 7KN + N) + T(2KF + NK(10+\Delta) + 2N + 1) + Q(L,K,N)$ \\
    \bottomrule
  \end{tabular}
\end{center}

\medskip\noindent
The next table shows some experimental data over Blocksworld and Gripper:
\begin{center}
  \begin{tabular}{lcrrrrrr@{}rrr}
    \toprule
              &        &     &     & \multicolumn{3}{c}{Input}        && \multicolumn{3}{c}{SAT Theory} \\
    \cmidrule{5-7}
    \cmidrule{9-11}
      problem & status & $K$ & $N$ & \#states & \#trans. & \#features && \#vars & \#clauses & Minisat \\
    \midrule
     Blocks04 &    SAT &   5 &   8 &      125 &      272 &         73 &&  20,721 &   472,560 &   1.63 \\
     Blocks04 &  UNSAT &   5 &   7 &      125 &      272 &         73 &&  18,635 &   449,815 & $>$250 \\

    \midrule
     Blocks05 &    SAT &   5 &   8 &      866 &    2,090 &        102 && 149,223 & 4,387,384 &  16.31 \\
     Blocks05 &  UNSAT &   5 &   7 &      866 &    2,090 &        102 && 134,273 & 4,216,157 & $>$250 \\

    \midrule
    Gripper03 &    SAT &   5 &  10 &       88 &      280 &         12 &&  21,116 &   264,029 &   0.28 \\
    Gripper03 &  UNSAT &   5 &   9 &       88 &      280 &         12 &&  19,376 &   242,080 &   1.43 \\

    \midrule
    Gripper04 &    SAT &   5 &  10 &      256 &      896 &         14 &&  64,236 &   854,955 &   5.00 \\
    Gripper04 &  UNSAT &   5 &   9 &      256 &      896 &         14 &&  58,968 &   785,686 &   6.86 \\

    \midrule
    Gripper05 &    SAT &   5 &  10 &      704 &    2,624 &         16 && 182,636 & 2,546,913 &   2.47 \\
    Gripper05 &  UNSAT &   5 &   9 &      704 &    2,624 &         16 && 167,704 & 2,345,740 &  39.16 \\

    \midrule
    Gripper06 &    SAT &   5 &  10 &    1,856 &    7,232 &         18 && 493,685 & 7,518,671 &  39.25 \\
    Gripper06 &  UNSAT &   5 &   9 &    1,856 &    7,232 &         18 && 453,409 & 6,931,130 & 110.45 \\

    \bottomrule
  \end{tabular}
\end{center}


\section{Preliminaries}

The sample consists of transitions of the form $t=\tup{s,a,s'}$ that
are assumed to come from one (**** CHECK: or more than one? ****) instances
of a generalized problem.

The $K$ features selected from $\F$ define an abstract space given by the
$2^K$ boolean valuations of the $K$ features. For boolean features, such a
valuation contains the value of the feature. For numerical features, the
boolean value denotes whether the numerical feature is equal to or bigger
than zero.
An abstract state represents a set of concrete states in the sample.
Abstract actions look like a regular STRIPS action on the abstract space.
An abstract action $\tilde a$ is made of a precondition and an effect, both
being partial valuations of the $K$ features. For boolean features $f$, $f$
(resp.\ $\neg f$) appears in the precondition iff $\tilde a$ requires $f$
to hold (resp.\ not hold) in the abstract state, and $f$ (resp.\ $\neg f$)
appears in the effect if $\tilde a$ makes $f$ true (resp.\ false).
For numerical effects, positive (resp.\ negative) $f$-literals in the
precondition stand for the condition $f>0$ (resp.\ $f=0$), and positive
(resp.\ negative) $f$-literals in the effect stand for an increment
(resp.\ decrement) of $f$.
Hence, whether the feature $f$ is boolean or numeric, we can specify
the precondition or effect of an abstract action on $f$ with a boolean
$f$-literal.

Once abstract actions are defined over the abstract space, verifying
soundness is just a matter of checking whether for each abstract
transition, there is a matching concrete transition.
That is, let us suppose that the abstract state $\tilde s$ is mapped
by the abstract action $\tilde a$ into the abstract state $\tilde s'$.
The abstract action $\tilde a$ is made of a precondition and effect.
The precondition must hold at $\tilde s$ while the effect maps $\tilde s$
into $\tilde s'$.
We say that $\tilde a$ is sound (wrt $\M$) iff for each concrete state
$s$ on which the precondition of $\tilde a$ holds, there is transition
$t=\tup{s,a,s'}$ in $\M$ that ``matches'' the abstract transition
$\tilde t=\tup{\tilde s,\tilde a,\tilde s'}$.
That is, $s$ is represented by $\tilde s$ (i.e.\ for each of the $K$
features $f$, the boolean value of $f$ in $s'$ and $\tilde s'$ coincide),
$s'$ is represented by $\tilde s'$, and the selected numerical features
change in the transition $t$ as indicated by the abstact action
$\tilde a$: if $f$ is increased (resp.\ decreased) by $\tilde a$,
then $f$ increases (resp.\ decreases) along the transition $t$.

Completeness is the opposite. The abstract model is complete if for each
transition $t=\tup{s,a,s'}$ there is an abstract transition
$\tilde t=\tup{\tilde s,\tilde a,\tilde s'}$ that matches $t$.

Let $\sim$ be the equivalance relation over concrete states defined by
the $K$ selected features: $s\sim s'$ iff for each selected feature $f$,
the boolean value of $f$ at $s$ and $s'$ are the same. For state $s$
and abstract state $\tilde s$, we write $s\sim\tilde s$ when the boolean
value of the $K$ selected features coincide in $s$ and $\tilde s$.
Likewise, we denote with $[s]$ and $[\tilde s]$ the equivalence class
for state $s$ and the set $\{s:s\sim\tilde s\}$ respectively, and
we write $t\sim\tilde t$, for transition $t$ and abstract transition $\tilde t$,
when $t$ matches $\tilde t$.


\section{Encoding}

The input to the reduction consists of the matrix $M$ and the parameters
$K$ and $N$. We assume that the set of feature $\F$ is enumerated as
$f_0, f_1,\ldots$ in such a way that the boolean features appear only
after the numerical features, and that the first boolean feature $f_i$
has index $i$ that is a power of 2. The gap between the last numerical
feature and the first boolean feature is assumed to be filled with dummy
features.
With this encoding, checking whether a selected feature $f_i$ is numerical
just amounts to check whether its index is less than the index of the first
boolean feature.

For each $j\in K$, let $F^j$ be a multi-valued (meta-)variable with domain
$F$ that denotes the $j$-th feature selected from $\F$.
Each $F^j$ variable is encoded with $\lceil \log_2 F \rceil$ propositional
variables $F^j_k$ where $F$ is the number of features in $\F$.
To ensure different features are selected, it is enough that $F^i<F^j$ for
$i<j$.

In the construction of the abstract model, we only consider abstract actions
that put a precondition on the features affected by the action; i.e., an
abstract action cannot affect a feature for which there is no precondition,
but it may put a precondition on an unaffected feature.

Given $K$ selected features, there are at most $9^K$ different abstract actions.
Observe that each abstract action is the result of selecting a subset of affected
features from the $K$ features and considering the ways in which they can
be affected (2 ways for each feature), and selecting a subset of features
for preconditions and considering the ways the can appear as such (2 ways
for each feature).\footnote{The bound is not tight because we will require
  $f>0$ in precondition of actions that decrement $f$.}
Hence, a bound on the total number of abstract actions is
\begin{alignat}{1}
  \sum_{k=1}^K \binom{K}{k} 2^k \times \sum_{k=0}^{K} \binom{K}{k} 2^k\ =\ 3^K \times (3^K - 1)\ \leq\ 9^K \,.
\end{alignat}
Since $K$ is constant, this number is also constant.

Let $A^j$, for $j\in\{0,\ldots,N-1\}$, denote the $j$-th abstract action.
Instead of using an exact enumeration on the different abstract actions,
we consider the following loose enumeration.
As mentioned above, each abstract action is the result of selecting a subset
of the selected features as affected, putting an effect over such features,
and then adding preconditions.
Thus, each abstract action can be specified with $K$ bits to selected the
subset of affected features, $K$ bits for speciyfing the effect on each affected
feature, $K$ additional bits that select preconditions, and a final
block of $K$ bits that specify the value of the preconditions.
In total, we need $4K$ bits to specify an abstract action.
We use propositional variables $A^j_{i,k}$ for $j\in N$, $i\in\{1,\ldots,4\}$,
and $k\in\{0,\ldots,K-1\}$ to specify the abstract action $A^j$ where
\begin{enumerate}[--]\denselist
  \item the group $A^j_{1,k}$ selects the affected features,
  \item the group $A^j_{2,k}$ sets the effect for the selected features,
  \item the group $A^j_{3,k}$ selects the features for preconditions, and
  \item the group $A^j_{4,k}$ sets the precondition for the additional features.
\end{enumerate}
We assume that the precondition $f>0$ must be present when $f$ is decremented,
and we require that each abstract action affects at least one feature.
These restrictions are enforced with:
\begin{alignat}{1}
  &A^j_{2,k} \rightarrow A^j_{1,k} \,, \\
  &A^j_{4,k} \rightarrow A^j_{3,k} \,, \\
  &A^j_{1,k} \land \neg A^j_{2,k} \land [\text{$F^k$ is numeric}] \rightarrow A^j_{4,k} \,, \\
  &A^j_{1,0} \lor A^j_{1,1} \lor \cdots \lor A^j_{1,K-1} \,.
\end{alignat}
The total number of such clauses is $3NK + N$. On the other hand, symmetries
are broken among abstract actions with a weak ordering on them and by
requiring $A^0\leq A^1 \leq \cdots \leq A^{N-1}$. Abstract actions are ordered
using the $K$ bits for selected the affected features, for each $j\in\{1,\ldots,N-1\}$
and $k\in\{0,\ldots,K-1\}$:
\begin{alignat}{1}
  &\neg A^j_{1,0} \land \cdots \land \neg A^j_{1,k-1} \land A^j_{1,k} \rightarrow A^{j-1}_{1,0} \lor \cdots \lor A^{j-1}_{1,k} \,.
\end{alignat}
The number of clauses in (2)--(8) is $3NK + N + (N-1)K = N(1 + 4K) - K$.

\medskip\noindent
The clauses needed for the $F^k_t$ variables establish that no dummy
or inexistent feature is selected, and enforce the strict ordering
among features. The first group corresponds to the following clauses
where $d$ is an index of a dummy or inexistent feature:
\begin{alignat}{1}
  &\bigvee_{t:bit(t,d)} \neg F^k_t \ \lor \bigvee_{t:\neg bit(t,d)} F^k_t
\end{alignat}
where $bit(t,d)$ is a boolean constant for the $t$-th bit in the binary
expansion of $d$.  The ordering among selected featues is enforced with
\begin{alignat}{1}
  &df(k,0) \lor df(k,1) \lor \cdots \lor df(k,L)
\end{alignat}
for $k\in\{0,\ldots,K-2\}$ and $L=\log_2(F)$, and the boolean
variables $df(k,t)$ satisfy
\begin{alignat}{1}
  &df(k,t) \rightarrow \neg F^k_t \,, \\
  &df(k,t) \rightarrow \neg F^{k+1}_t \,, \\
  &df(k,t) \land F^k_i \rightarrow F^{k+1}_i \qquad \text{for $i\in\{t+1,\ldots,L-1\}$} \,, \\
  &df(k,t) \land \neg F^{k+1}_i \rightarrow F^k_i \qquad \text{for $i\in\{t+1,\ldots,L-1\}$} \,.
\end{alignat}
The total number of clauses in (7)--(12) is $K(D+U) + (K-1)(1 + 2L + L(L-1))$
where $D$ and $U$ are the number of dummy and inexistent features respectively.
The total number of clauses in (2)--(12) is
$Q(L,K,N)=N(1+4K) + K(D+U) + (K-1)(L^2 + L) - 1$.


\subsection{Value of selected features at states}

We define the variable $\phi(k,i)$ that tells whether the selected feature $F^k$
holds at state $s_i$ in the sample; $\phi(k,i)$ can be defined in two equivalent
ways:
\begin{alignat}{1}
  \label{eq:phi:1}
  \phi(k,i)\
    &\equiv\ \bigwedge_{\ell \in \F} [F^k = f_\ell] \rightarrow (m_{\ell i} > 0) \
     \equiv\ \bigwedge_{\ell \in \F: m_{\ell i}=0} \neg [F^k = f_\ell] \
     \equiv\ \bigwedge_{\ell \in \F: m_{\ell i}=0} \neg \biggl( \bigwedge_t F^k_t \leftrightarrow bit(t,\ell) \biggr) \\
  \label{eq:phi:2}
  \phi(k,i)\
    &\equiv\ \bigvee_{\ell \in \F} [F^k = f_\ell] \land (m_{\ell i} > 0) \
     \equiv\ \bigvee_{\ell \in \F: m_{\ell i}>0} [F^k = f_\ell] \
     \equiv\ \bigvee_{\ell \in \F: m_{\ell i}>0} \biggl( \bigwedge_t F^k_t \leftrightarrow bit(t,\ell) \biggr)
\end{alignat}
where $[F^k=f_\ell]$ denotes the formula that says the $k$-th selected feature is
$f_\ell$.
Therefore, for each $k\in K$ and $i\in\M$, the variable $\phi(k,i)$ is defined
with the following implications:
\begin{alignat}{1}
  \phi(k,i)\ &\rightarrow\ \biggl(\bigvee_{t:bit(t,\ell)} \neg F^k_t\biggr) \lor \biggl(\bigvee_{t:\neg bit(t,\ell)} F^k_t\biggr) \,,
               \quad \text{for each $\ell$ such that $m_{\ell i}=0$,} \\
  \phi(k,i)\ &\leftarrow\  \biggl(\bigwedge_{t:bit(t,\ell)} F^k_t\biggr) \land \biggl(\bigwedge_{t:\neg bit(t,\ell)} \neg F^k_t\biggr) \,,
               \quad \text{for each $\ell$ such that $m_{\ell i}>0$.}
\end{alignat}
These account for exactly $SKF$ clauses where $S$ is the number of states in $\M$
and $F$ is the number of features in $\F$.
Likewise, we define the variable $\phi^*(k,i)$ that tells whether the selected
feature $F^j$ \emph{does not} hold at $s_i$:
\begin{alignat}{1}
  \label{eq:phi*:1}
  \phi^*(k,i)\
    &\equiv\ \bigwedge_{\ell \in \F} [F^k = f_\ell] \rightarrow (m_{\ell i} = 0) \
     \equiv\ \bigwedge_{\ell \in \F: m_{\ell i}>0} \neg [F^k = f_\ell] \
     \equiv\ \bigwedge_{\ell \in \F: m_{\ell i}>0} \neg \biggl( \bigwedge_t F^k_t \leftrightarrow bit(t,\ell) \biggr) \\
  \label{eq:phi*:2}
  \phi^*(k,i)\
    &\equiv\ \bigvee_{\ell \in \F} [F^k = f_\ell] \land (m_{\ell i} = 0) \
     \equiv\ \bigvee_{\ell \in \F: m_{\ell i}=0} [F^k = f_\ell] \
     \equiv\ \bigvee_{\ell \in \F: m_{\ell i}=0} \biggl( \bigwedge_t F^k_t \leftrightarrow bit(t,\ell) \biggr)
\end{alignat}
with the following $SKF$ clauses:
\begin{alignat}{1}
  \phi^*(k,i)\ &\rightarrow\ \biggl(\bigvee_{t:bit(t,\ell)} \neg F^k_t\biggr) \lor \biggl(\bigvee_{t:\neg bit(t,\ell)} F^k_t\biggr) \,,
                \quad \text{for each $\ell$ such that $m_{\ell i}>0$,} \\
  \phi^*(k,i)\ &\leftarrow\  \biggl(\bigwedge_{t:bit(t,\ell)} F^k_t\biggr) \land \biggl(\bigwedge_{t:\neg bit(t,\ell)} \neg F^k_t\biggr) \,,
                \quad \text{for each $\ell$ such that $m_{\ell i}=0$.}
\end{alignat}


\subsection{Soundness}

We express soundness with the following formulas, one for each state $s$ in the
sample $\M$, and for each $j\in\{0,\ldots,N-1\}$:
\begin{alignat}{1}
  \label{eq:soundness}
  app(A^j,[s]) \rightarrow \textstyle\bigvee_{t=\tup{s,a,s'}\in\M} t \sim \tup{[s],A^j,res(A^j,[s])}
\end{alignat}
where $app(A^j,[s])$ means that the abstract action denoted by $A^j$ is applicable
at the abstract state $\tilde s$ such that $s\sim\tilde s$, $res(A^j,[s])$ denotes
the abstract state that results of applying $A^j$ in $\tilde s$, and
$\tilde t = \tup{[s],A^j,res(A^j,[s]}$ is the abstract transition from $\tilde s$
to $\tilde s'=res(A^j,[s])$ induced by $A^j$.

For $s_i=s$, the formula $app(A^j,[s])$ is captured with:
\begin{alignat}{1}
  app(A^j,[s])\
    &\equiv\ \bigwedge_{k} \biggl(A^j_{4,k} \rightarrow \phi(k,i)\biggr)\ \land\
             \bigwedge_{k} \biggl(A^j_{3,k} \land \neg A^j_{4,k} \rightarrow \phi^*(k,i)\biggr)\
     \equiv\ \bigwedge_{k} B_1(j,i,k) \land B_2(j,i,k)
\end{alignat}
where the auxiliary variables $B_1(j,i,k)$ and $B_2(j,i,k)$ are given by:
\begin{alignat}{1}
  &B_1(j,i,k) \land A^j_{4,k} \rightarrow \phi(k,i)\,, \\
  &\neg A^j_{4,k} \rightarrow B_1(j,i,k)\,, \\
  &\phi(k,i) \rightarrow B_1(j,i,k)\,, \\[1em]
  &B_2(j,i,k) \land A^j_{3,k} \land \neg A^j_{4,k} \rightarrow \phi^*(k,i)\,, \\
  &\neg A^j_{3,k} \rightarrow B_2(j,i,k)\,, \\
  &A^j_{4,k} \rightarrow B_2(j,i,k)\,, \\
  &\phi^*(k,i) \rightarrow B_2(j,i,k)\,.
\end{alignat}
In total, we need $7NSK$ clauses to define the auxiliary $B$ variables.

Let us focus on the consequent of the implication that defines soundness. We need
an expression for $t \sim \tup{[s],A^j,res(A^j,[s])}$ where $t=\tup{s,a,s'}$ is
a transition in $\M$, $A^j$ is an abstract action, and $res(A^j,[s])$ denotes
the abstract state that results of applying $A^j$ on $[s]$.
By definition, $\tup{s,a,s'}\sim\tup{\tilde s,\tilde a,\tilde s'}$ iff
1)~$s\sim\tilde s$, 2)~$s'\sim\tilde s'$, and 3)~the numeric features change
in both transitions in a compatible way.
In our case, $t=\tup{s_i,a,s_\ell}$ and $\tilde t=\tup{[s_i],A^j,res(A^j,[s_i])}$
and (1) is always satisfied.
For (2), we use the auxiliary variable $RES(i,\ell,j)$ to encode $s_\ell\sim res(A^j,[s_i])$:
\begin{alignat}{1}
  RES(i,\ell,j)\
    &\equiv\  \bigwedge_{k} \biggl(A^j_{2,k} \rightarrow \phi(k,\ell)\biggr)\ \land\
              \bigwedge_{k} \biggl(A^j_{1,k} \land \neg A^j_{2,k} \land \neg [\text{$F^k$ is numeric}] \rightarrow \phi^*(k,\ell)\biggr)\ \land \\
    &\quad\ \ \bigwedge_{k} \biggl(\neg A^j_{1,k} \land \phi(k,i) \rightarrow \phi(k,\ell)\biggr)\ \land\
              \bigwedge_{k} \biggl(\neg A^j_{1,k} \land \phi^*(k,i) \rightarrow \phi^*(k,\ell)\biggr)\,.
\end{alignat}
Since $RES(i,\ell,j)$ only appears in the consequent of the implications defining soundness,
it is enough to enforce the following implications:
\begin{alignat}{1}
  &RES(i,\ell,j) \land A^j_{2,k} \rightarrow \phi(k,\ell)\,, \\
  &RES(i,\ell,j) \land A^j_{1,k} \land \neg A^j_{2,k} \land \neg [\text{$F^k$ is numeric}] \rightarrow \phi^*(k,\ell)\,, \\
  &RES(i,\ell,j) \land \neg A^j_{1,k} \land \phi(k,i) \rightarrow \phi(k,\ell)\,, \\
  &RES(i,\ell,j) \land \neg A^j_{1,k} \land \phi^*(k,i) \rightarrow \phi^*(k,\ell)\,.
\end{alignat}
There are 4 implications for each transition $t=\tup{s_i,a,s_\ell}$ in $\M$,
each $j\in\{0,\ldots,N-1\}$, and each $k\in\{0,\ldots,K-1\}$.

The final requirement to enforce $t\sim\tilde t$ is that the numerical features
change in both transitions in a compatible manner. For each numerical feature $f$,
if $f$ is increased (resp.\ decreased) by $A^j$, then $f$ increases (resp.\ decreases)
in the transition from $s$ to $s'$. We use auxiliary variables $INC(k,i,\ell)$
(resp.\ $DEC(k,i,\ell)$) to denote that feature $k$ increases (resp.\ decreases)
in the transition $\tup{s_i,a,s_\ell}$ in $\M$.\footnote{**** CHECK: Can we have
  two transitions with equal source and destination but with different actions?
  Does it matter? It seems that it doesn't matter since whether $f$ increases
  or decreases in a transition only depent of the values of $f$ at the states.
  ****}
The $INC(k,i,\ell)$ variable can be expressed in two different ways:
\begin{alignat}{1}
  INC(k,i,\ell)\ &\equiv\ \bigwedge_{j\in\F} [F^k = f_j] \rightarrow (m_{ji} < m_{j\ell})\
                  \equiv\ \bigwedge_{j\in\F:m_{ji}\geq m_{j\ell}} \neg [F^k = f_j] \\
                 &\equiv\ \bigwedge_{j\in\F:m_{ji}\geq m_{j\ell}} \neg \biggl(\bigwedge_t F^k_t \leftrightarrow bit(t,j)\biggr) \,, \\
  INC(k,i,\ell)\ &\equiv\ \bigvee_{j\in\F} [F^k = f_j] \land (m_{ji} < m_{j\ell})\
                  \equiv\ \bigvee_{j\in\F:m_{ji}<m_{j\ell}} [F^k = f_j] \\
                 &\equiv\ \bigvee_{j\in\F:m_{ji}<m_{j\ell}} \biggl(\bigwedge_t F^k_t \leftrightarrow bit(t,j)\biggr) \,.
\end{alignat}
We therefore use the following implications to define $INC(k,i,\ell)$:
\begin{alignat}{1}
  INC(k,i,\ell)\ &\rightarrow\ \biggl(\bigvee_{t:bit(t,j)} \neg F^k_t\biggr) \lor \biggl(\bigvee_{t:\neg bit(t,j)} F^k_t\biggr) \,,
                 \quad \text{for each $j$ such that $m_{ji} \geq m_{j\ell}$,} \\
  INC(k,i,\ell)\ &\leftarrow\  \biggl(\bigwedge_{t:bit(t,j)} F^k_t\biggr) \land \biggl(\bigwedge_{t:\neg bit(t,j)} \neg F^k_t\biggr) \,,
                 \quad \text{for each $j$ such that $m_{ji} < m_{j\ell}$.}
\end{alignat}
The $DEC(k,i,\ell)$ variables are defined analogously with the implications:
\begin{alignat}{1}
  DEC(k,i,\ell)\ &\rightarrow\ \biggl(\bigvee_{t:bit(t,j)} \neg F^k_t\biggr) \lor \biggl(\bigvee_{t:\neg bit(t,j)} F^k_t\biggr) \,,
                 \quad \text{for each $j$ such that $m_{ji} \leq m_{j\ell}$,} \\
  DEC(k,i,\ell)\ &\leftarrow\  \biggl(\bigwedge_{t:bit(t,j)} F^k_t\biggr) \land \biggl(\bigwedge_{t:\neg bit(t,j)} \neg F^k_t\biggr) \,,
                 \quad \text{for each $j$ such that $m_{ji} > m_{j\ell}$.}
\end{alignat}
Therefore, there are $2TK$ such auxiliaries variables that are defined with $2TFK$ implications.
The following auxiliary variables capture $\tup{s_i,a,s_\ell}\sim\tup{[s_i],A^j,res(A^j,[s_i)}$:
\begin{alignat}{1}
  MATCH(i,\ell,j)\ &\equiv\ RES(i,\ell,j)\ \land \\
                   &\quad\ \, \bigwedge_k \biggl([\text{$F^k$ is numeric}] \rightarrow (A^j_{2,k} \leftrightarrow INC(k,i,\ell))\biggr)\ \land \\
                   &\quad\ \, \bigwedge_k \biggl(A^j_{1,k} \land [\text{$F^k$ is numeric}] \rightarrow (\neg A^2_{2,k} \leftrightarrow DEC(k,i,\ell))\biggr) \,.
\end{alignat}
As these variables only appear in the consequent of implications, it is sufficient to only
enforce:
\begin{alignat}{1}
  &MATCH(i,\ell,j) \rightarrow RES(i,\ell,j) \,, \\
  &MATCH(i,\ell,j) \land [\text{$F^k$ is numeric}] \land A^j_{2,k} \rightarrow INC(k,i,\ell) \,, \\
  &MATCH(i,\ell,j) \land [\text{$F^k$ is numeric}] \land INC(k,i,\ell) \rightarrow A^j_{2,k} \,, \\
  &MATCH(i,\ell,j) \land [\text{$F^k$ is numeric}] \land A^j_{1,k} \land \neg A^j_{2,k} \rightarrow DEC(k,i,\ell) \,, \\
  &MATCH(i,\ell,j) \land [\text{$F^k$ is numeric}] \land DEC(k,i,\ell) \rightarrow A^j_{1,k} \,, \\
  &MATCH(i,\ell,j) \land [\text{$F^k$ is numeric}] \land DEC(k,i,\ell) \rightarrow \neg A^j_{2,k} \,.
\end{alignat}
There are $TN + 5TNK$ such implications.

Finally, soundness is enforced with $SN$ implications, one for each state $s_i$ and $j\in\{0,\ldots,N-1\}$:
\begin{alignat}{1}
  \biggl(\bigwedge_k B_1(j,i,k) \land B_2(j,i,k)\biggr)\ \rightarrow
  \bigvee_{t=\tup{s_i,a,s_\ell}} MATCH(i,\ell,j) \,.
\end{alignat}


\subsection{Completeness}

For each transition $t=\tup{s,a,s'}$ in $\M$, there is an abstract transition $\tilde t$ such that $t\sim\tilde t$;
that is, for each transition $t=\tup{s_i,a,s_\ell}$ in $\M$:
\begin{alignat}{1}
  \label{eq:completeness}
  \textstyle\bigvee_{j} app(A^j,[s_i]) \land t \sim \tup{[s_i],A^j,res(A^j,[s_i])} \,.
\end{alignat}
Since $MATCH(i,\ell,j)$ implies $\tup{s_i,a,s_\ell}\sim\tup{[s_i],A^j,res(A^j,[s_i])}$,
completeness can be capture with
\begin{alignat}{1}
  \label{eq:completeness}
  \textstyle\bigvee_{j} APPMATCH(i,\ell,j) \,,
\end{alignat}
one for each transition $\tup{s_i,a,s_\ell}$ in $\M$, where the auxiliary variables
$APPMATCH(i,\ell,j)$ satisfy
\begin{alignat}{1}
  &APPMATCH(i,\ell,j) \rightarrow MATCH(i,\ell,j) \,, \\
  &APPMATCH(i,\ell,j) \rightarrow B_1(j,i,k) \,, \\
  &APPMATCH(i,\ell,j) \rightarrow B_2(j,i,k)
\end{alignat}
for $k=0,\ldots,K-1$.

\end{document}

